%%%%%%%%%%%%%%%%%%%%%%%%%%%%%%%%%%%%%%%%%%%%%%%%%%%%%%%%%%%%%%%

% Set up document

\documentclass{beamer}
% For themese see https://hartwork.org/beamer-theme-matrix/
%\usetheme{Warsaw}
\usecolortheme{beetle}
\setbeamersize{text margin left=5mm,text margin right=5mm}

% Used to create a section slide between section
\AtBeginSection[]{
  \begin{frame}
  \vfill
  \centering
  \begin{beamercolorbox}[sep=8pt,center,shadow=true,rounded=true]{title}
    \usebeamerfont{title}\insertsectionhead\par%
  \end{beamercolorbox}
  \vfill
  \end{frame}
}

% Remove default navigation symbols and add just  page number
\setbeamertemplate{navigation symbols}{} % Clear default navigation
\addtobeamertemplate{navigation symbols}{}{%
    \usebeamerfont{footline}%
    \usebeamercolor[fg]{footline}%
    \hspace{1em}%
    \insertframenumber/\inserttotalframenumber
}


%%%%%%%%%%%%%%%%%%%%%%%%%%%%%%%%%%%%%%%%%%%%%%%%%%%%%%%%%%%%%%%

% Title page text

\title{What would other emergency stroke teams do?}
\subtitle{Using explainable machine learning to understand variation in thrombolysis practice}


\author{Kerry Pearn\inst{1}, Michael Allen\inst{1,3}, Anna Laws\inst{1}, Richard Everson\inst{3}, Martin James\inst{1,2} }
\institute{\inst{1}University of Exeter Medical School \inst{2}Royal Devon University Healthcare NHS Foundation Trust \inst{3}University of Exeter Institute of Data Science and Artificial Intelligence}

%\institute{Overleaf}
\date{January 2023}


\begin{document}

%%%%%%%%%%%%%%%%%%%%%%%%%%%%%%%%%%%%%%%%%%%%%%%%%%%%%%%%%%%%%%%

%\frame{\titlepage}

\begin{frame}
\titlepage
\end{frame}

%%%%%%%%%%%%%%%%%%%%%%%%%%%%%%%%%%%%%%%%%%%%%%%%%%%%%%%%%%%%%%%


\begin{frame}
\frametitle{Thrombolysis rates vary .... a lot}
Here we show the range of thrombolysis use across the 132 acute stroke centres in England and Wales. The NHS target is that 20\% of patients should be receiving thrombolysis.
\begin{center}
\includegraphics[width=1.0\textwidth]{./images/thrombolysis_hist}
\end{center}

How much of this variation is due to differences in decisions hospitals make on who they would give thrombolysis to?
\end{frame}

%%%%%%%%%%%%%%%%%%%%%%%%%%%%%%%%%%%%%%%%%%%%%%%%%%%%%%%%%%%%%%%

\begin{frame}
\frametitle{Breaking down the emergency stroke pathway into key steps}
\begin{center}
\includegraphics[width=1.0\textwidth]{./images/pathway}
\end{center}
We can model key changes to pathway:
\begin{small}
\begin{itemize}
    \item What if the pathway were faster?
    \item What if hospital determined the stroke onset time in more patients?
    \item What if clinical decision-making was like that of \emph{benchmark} hospitals? (Predict what treatment a patient would receive at other hospitals).
\end{itemize}
\end{small}
\footnotesize{We model these changes with a hospital's own patient population, to allow for inter-hospital variation in patient population characteristics.}
\end{frame}

%%%%%%%%%%%%%%%%%%%%%%%%%%%%%%%%%%%%%%%%%%%%%%%%%%%%%%%%%%%%%%%

\begin{frame}
\frametitle{Machine learning overview}
\begin{center}
\includegraphics[width=0.90\textwidth]{./images/ml_model_high_level}
\end{center}

\footnotesize
Machine learning (and nearly all \emph{artificial intelligence}) is based on the simple principle of recognising similarity to what has been seen before.
\vspace{3mm}

We accessed 240,000 emergency stroke admissions in England and Wales over three years. For machine learning we use 88,000 patients who all arrive within 4 hours of known stroke onset.
\end{frame}


%%%%%%%%%%%%%%%%%%%%%%%%%%%%%%%%%%%%%%%%%%%%%%%%%%%%%%%%%%%%%%%

\begin{frame}
\frametitle{A \emph{Black Box} model}
When we don't know how a model is making predictions we call it a \emph{Black Box} model.
\vspace{4mm}
\begin{center}
\includegraphics[width=0.90\textwidth]{./images/black_box}
\end{center}
\end{frame}


%%%%%%%%%%%%%%%%%%%%%%%%%%%%%%%%%%%%%%%%%%%%%%%%%%%%%%%%%%%%%%%

\begin{frame}
\frametitle{Simplifying our model}

In order to simplify the model, to make it easier to explain, we reduced the number of patient features from 60 to 10, with almost no loss of model accuracy.

\small
\begin{itemize}
    \item \emph{Arrival-to-scan time}: Time from arrival at hospital to scan (mins)
    \item \emph{Infarction}: Stroke type (infarction or haemorrhage)
    \item \emph{Stroke severity}: Stroke severity (NIHSS) on arrival (ranges from 0 to 42)
    \item \emph{Precise onset time}: Onset time is known precisely
    \item \emph{Prior disability level}: Disability level (modified Rankin Scale) before stroke
    \item \emph{Stroke team}: Stroke team attended
    \item \emph{Use of AF anticoagulants}: Use of atrial fibrillation anticoagulant
    \item \emph{Onset-to-arrival time}: Time from onset of stroke to arrival at hospital
    \item \emph{Onset during sleep}: Did stroke occur in sleep?
    \item \emph{Age}: Age (as middle of 5 year age bands)
\end{itemize}
\end{frame}

%%%%%%%%%%%%%%%%%%%%%%%%%%%%%%%%%%%%%%%%%%%%%%%%%%%%%%%%%%%%%%%

\begin{frame}
\frametitle{SHAP values - looking into the \emph{Black Box}}

SHAP (SHapley Additive exPlanations) values show us the contribution of each patient feature, even in a \emph{Black Box} model.

\vspace{2mm}
These are found by running predictions lots of times with only part of the data for each patient available each time.


\begin{center}
\includegraphics[width=0.75\textwidth]{./images/xgb_waterfall_low_probability_2}
\end{center}
\end{frame}

\begin{frame}
\frametitle{What general patterns did we see?}

The  model revealed that the odds of receiving thrombolysis:
\vspace{1mm}
\begin{itemize}
    \item Reduced 20 fold over the first 100 minutes of arrival-to-scan time
    \item Varied 30 fold depending on stroke severity, with lowest thrombolysis use at low or very high stroke severities
    \item Reduced 3 fold when the stroke onset time was not precisely known
    \item Fell 5 fold with increasing pre-stroke disability
    \item Varied 15 fold between hospitals
\end{itemize}

\vspace{2mm}

The majority of the variation in thrombolysis use between hospitals may be explained by differences in the hospitals’ willingness to use the treatment, rather than the characteristics of the patients they treated.

\vspace{2mm}

Compared with hospitals with higher thrombolysis use, hospitals with lower use were particularly less likely to give thrombolysis to patients with milder strokes, prior disability, or patients with imprecise onset time.

\end{frame}

%%%%%%%%%%%%%%%%%%%%%%%%%%%%%%%%%%%%%%%%%%%%%%%%%%%%%%%%%%%%%%%

\begin{frame}
\frametitle{Viewing the SHAP result - 1}
The previous observations on what affects the odds of receiving thrombolysis, came from these SHAP plots.
\begin{center}
\includegraphics[width=1.0\textwidth]{./images/shap_violin_1}
\end{center}

\scriptsize
SHAP effects: 

$\pm1$: Odds change $\pm3$ fold

$\pm2$: Odds change $\pm7$ fold

$\pm3$: Odds change $\pm20$ fold

$\pm4$: Odds change $\pm55$ fold

$\pm5$: Odds change $\pm150$ fold
    



\end{frame}

\begin{frame}
\frametitle{Viewing the SHAP result - 2}
The previous observations on what affects the odds of receiving thrombolysis, came from these SHAP plots.
\begin{center}
\includegraphics[width=1.0\textwidth]{./images/shap_violin_2}
\end{center}

\scriptsize
SHAP effects: 

$\pm1$: Odds change $\pm3$ fold

$\pm2$: Odds change $\pm7$ fold

$\pm3$: Odds change $\pm20$ fold

$\pm4$: Odds change $\pm55$ fold

$\pm5$: Odds change $\pm150$ fold
\end{frame}




%%%%%%%%%%%%%%%%%%%%%%%%%%%%%%%%%%%%%%%%%%%%%%%%%%%%%%%%%%%%%%%

\begin{frame}
\frametitle{Artificial patients}

We can \emph{construct} artificial patients:
\vspace{1mm}
\begin{footnotesize}
\begin{itemize}
    \item Arrival-to-scan time = 15 minutes 
    \item Stroke type  = Infarction
    \item Stroke severity (NIHSS) = 15
    \item Precise onset time = Yes
    \item Prior disability level = None
    \item Use of AF anticoagulants = No
    \item Onset-to-arrival time = 60 minutes
    \item Onset during sleep = No
    \item Age = 72
\end{itemize}
\end{footnotesize}

\vspace{2mm}

99\% hospitals would be expected to give this patient thrombolysis.

\vspace{2mm}

If we change stroke severity to mild (NIHSS=5), and make the stroke onset time not precise, 35\% hospitals would be expected to give this patient thrombolysis.

\end{frame}

\begin{frame}{Summary}

\begin{footnotesize}
    


Stroke is a common cause of adult disability. Most strokes (about four out of five) are caused by a blood clot in the brain, and have the potential to be treated with clot-busting drugs to break up the blood clot that is causing their stroke - this is called thrombolysis. Use of thrombolysis is only about half of what the NHS long-term plan suggests it should be, and varies a lot between hospitals.

\vspace{3mm}

We used explainable machine learning to investigate which patients receive thrombolysis at each hospital. To do this we used data on nearly 90,000 patients who arrived at hospital with time still left to give thrombolysis (as thrombolysis must be given within four and a half hours after stroke onset).

\vspace{3mm}

We found that in patients arriving in time to receive thrombolysis, the use of thrombolysis ranged between hospitals from 7\% (1 in 14 patients) to 49\% (1 in 2 patients). The chance of receiving thrombolysis depended on the time it took to perform a brain scan, the severity of the stroke, whether the time of stroke onset was known precisely, and the level of disability the patient had before the stroke. But the most interesting finding was that the majority of the variation in thrombolysis usage between hospitals was due to the hospitals’ willingness and readiness to use the treatment, rather than any differences in patients between hospitals.

\end{footnotesize}

\end{frame}

\end{document}

